%
\documentclass[11pt]{article}

\newcommand{\name}{Wil Kock, Sophia Yakoubov }
\newcommand{\hw}{0}

%\pagestyle{headings}
\usepackage[dvips]{graphics,color}
\usepackage{amsfonts}
\usepackage{amssymb}
\usepackage{amsmath}
\usepackage{latexsym}
\usepackage{todonotes}
\setlength{\parskip}{1pc}
\setlength{\parindent}{0pt}
\setlength{\topmargin}{-3pc}
\setlength{\textheight}{9.5in}
\setlength{\oddsidemargin}{0pc}
\setlength{\evensidemargin}{0pc}
\setlength{\textwidth}{6.5in}

\newcommand{\head}{
\newpage
\noindent
\framebox{
	\vbox{
		CS655 (Networks) Programming Assignment \hw  \\ 
		\name \hfill \today
	}
}
\bigskip

}


\begin{document}

\head

\section{Program Documentation}

We implemented this assignment in Python, building off of the Sympy library.

\subsection{Routers}

We took a class-based approach to implementing our routers. 
We created a router superclass (in `store\_super.py'), FIFO and RR router classes subclassing it (in `store\_fifo.py' and `store\_rr.py' respectively), and a DRR router class subclassing the RR router class (in `store\_drr.py').

\subsection{Sources}

\subsection{Running Instructions}

\subsubsection{Installation}
At the command line run the following

./install.sh

If you are confrounted with permission denied run

chmod 755 install.sh

Next 
source virtualenv-1.9/ve_pa0/bin/activate

\section{Experimental Results}

\subsection{FIFO Router}

\subsection{RR Router}

\subsection{DRR Router}

\end{document}

