%!TEX root = report.tex

\section{Program Documentation}

We implemented this assignment in Python, and our architecture is built of off Simpy, a process-based discrete-event simulation library. Each source and the router is modelled as a processes while the transportation of packets are modelled as events. In our architecture we utilize a process interaction concept in Simpy known as Stores which allows the modelling of production and consumption of objects. 

\subsection{Routers}

We took a class-based approach to implementing our router. 
We created a router superclass (in `store\_super.py'), FIFO and RR router classes subclassing it (in `store\_fifo.py' and `store\_rr.py' respectively), and a DRR router class subclassing the RR router class (in `store\_drr.py').

\subsection{Sources}

Simpy provides another process interaction called Containers which allow the modelling of quantities of similar discrete objects. In order to limit the number of packets used in an experiment we use a container and fill it with the number of packets we want to run. Each source will then decrement the value in this container each time it creates a packet. When The container is empty, sources will no longer be able to create packets and once the router transmits the last packet the simulation will terminate.

\begin{equation}
\sum 
\end{equation}

\subsection{Running Instructions}

\subsubsection{Installation}
In order to use Simpy on the csa2.bu.edu server we must use virtualenv to be able to install Python modules without root privilege in a sandbox environment. We created an install script that will download, install and configure our test environment so the experiments can be run.

At the command line run the following

\begin{lstlisting}
./install.sh
\end{lstlisting}

If you are confronted with permission denied error you may need to change the permission on the file, at the terminal run:

\begin{lstlisting}
chmod 755 install.sh
\end{lstlisting}

\subsection{Statistics Collection}

All of our statistics are based on attributes stored on each packet.
When a packet is created, it is given a `length' attribute that indicates its length, in bits.
During the simulation, times at which the packet `arrives' at the router and `departs' from the router are also stored;
the router sets those time attributes for each packet object within its enqueuing and dequeuing methods.
The packet is then capable of reporting its latency (queueing delay) by computing the difference between its departure time and arrival time.

After the simulation, the collected information is used to calculate latency and throughput.
Average packet latency for a given source is computed as the sum of all of that source's packet's queueing delays, divided by the number of packets sent by that source.
Throughput for a given source is computed as the sum of all of that source's packet's lengths, divided by the difference between the last departure time and the first arrival time.
