%!TEX root = report.tex


\section{Program Documentation}

We implemented this assignment in Python, and our architecture is built of off Simpy, a process-based discrete-event simulation library. Each source and the router is modelled as a processes while the transportation of packets are modelled as events. 

\subsection{Queuing Algorithms}

In our architecture we utilize a process interaction concept in Simpy known as Stores which allows the modelling of production and consumption of objects. Stores allow objects to be put and get from processes. 
We abstract this concept to allow us to model the receiving and transmitting of packets by the router. 
When packets are sent to the router they are put in the Store, likewise when the router transmits a packet from a queue it is get from the Store.  
Fortunately Simpy allows for subclassing the Store class to provide your own implementation for how objects are put and get from the Store. 
We created three subclasses, for each queuing algorithms providing a robust flexible framework that allows different algorithms to be easily simulated. These classes are store\_fifo.py, store\_rr.py and store\_drr.py.

\subsection{Sources}

Simpy provides another process interaction concept called Containers which allow the modelling of quantities of similar discrete objects. In order to limit the number of packets used in an experiment we use a container and "fill it" with the number of packets we want to run. Each source will then decrement the value in this container each time it creates a packet. When The container is empty, sources will no longer be able to create packets and once the router transmits the last packet the simulation will terminate.

Each packet created by a source is represented by a packet object specifying its sequence number, source/destination addresses and the length of the packet. The source then transmits the packet which takes $L/r$ time, where $L$ is the length of the packet and $r$ is the transmission rate, the transmission delay is modelled by a time out for that processes. Once the simulation of transmission is complete the packet is put in the store signifying that the packet has reached the router.

\subsection{Logging}

In order to provide efficient debugging capabilities we created a logging utility that writes to a file the state of the queue during each event. This provides the user the ability to provide a full inspection to verify proper operation of the implemented queuing algorithm. An example of an output for FIFO is given below.

\begin{verbatim}
4(12186)@15755.0>
<<----------
976|2159|197|1076|885|519|745|590|3151|
<<----------
<<7(976)
----------<<
2159|197|1076|885|519|745|590|3151|177|
----------<<
\end{verbatim}

In the above example, source 4 (or 192.168.1.104) sends a packet of size 12186 at simulation time 15755. The state of the queue at that time is shown after. The packet of size 976 from source 7 is then transmitted by the router indicated by $<<$7(976). The state of the queue after 7s packet is transmitted is then show. We can see in the FIFO queue that the packet is no longer in the queue.

An example of how round-robin and deficit round-robin are logged are displayed below.

\begin{verbatim}
<<----------
0	6990
1	
3	
8	23|103|59|344|1148|641
9	239|996|433
4	7845|497
6	253|428|76|427|247|100|88
5	6571
7	733
10	5000|5000|5000
<<----------
<<9(239)
----------<<
0	6990
1	
3	
8	23|103|59|344|1148|641
9	996|433|1705
4	7845|497
6	253|428|76|427|247|100|88
5	6571
7	733
10	5000|5000|5000
----------<<
\end{verbatim}

In this example all of the source numbers are displayed on the left while each queue for the source is adjacent to its label. We can see that the router transmitted packet from source 9 of length 239 and at the next queue display this value has been popped off.



\subsection{Running Instructions}

\subsubsection{Installation}
In order to use Simpy on the csa2.bu.edu server we must use virtualenv to be able to install Python modules without root privilege in a sandbox environment. We created an install script that will download, install and configure our test environment so the experiments can be run.

At the command line run the following:

\begin{lstlisting}
./install.sh
\end{lstlisting}

If you are confronted with permission denied error you may need to change the permission on the file, at the terminal run:

\begin{lstlisting}
chmod 755 install.sh
\end{lstlisting}

Grab a coffee or tea and when you return the installation process should be complete. 

\subsubsection{Running an experiment}
It is important to note that the Python scripts \textbf{cannot be run directly}, they must utilize the virtualenv environment so the necessary modules will be imported. To make running experiments easier we have created bash scripts for each type of algorithm that includes the commands for running the code in the virtual environment. The main.py Python script will be executed in these bash scripts to run the experiment. This script has the following parameters:

-M [M] where [M] is an integer value specifying the total offered load, -x [X] where [X] is an integer value specifying the number of experiments to run, -n [N] where [N] is an integer value specifying how many packets to use in the experiment, -p is an optional flag indicating plots are to be generated and lastly the algorithm must be specified, one of --fifo, --rr, --drr, indicating FIFO, round robin and deficit round robin respectively.

An example of a typical command we used to run 10 experiments with 100,000 packets, total offered load of 1 for fifo algorithm and plot generation would be:

\begin{verbatim}
python main.py -x 10 -M 1 -n 100000 --fifo -p
\end{verbatim}

As previously mentioned we made it easier to run in the virtual environment by creating individual scripts to run. To run an experiment simulating the FIFO algorithm run the following at the command line:

\begin{lstlisting}
./exp_fifo.sh
\end{lstlisting}

To run an experiment simulating the round robin algorithm:

\begin{lstlisting}
./exp_rr.sh
\end{lstlisting}

And deficit round robin algorithm:

\begin{lstlisting}
./exp_drr.sh
\end{lstlisting}

If for any reason these commands fail for permission issues run chmod 755 and the file name was previously demonstrated. To test different parameters for the experiment, open the corresponding bash script with your favorite text editor and make the command line parameter changes to the main.py command. While the script runs it will output summary of results and at the end it will output a log file with the suffix .out.

\subsection{Statistics Collection}

All of our statistics are based on attributes stored on each packet.
When a packet is created, it is given a `length' attribute that indicates its length, in bits.
During the simulation, times at which the packet `arrives' at the router and `departs' from the router are also stored;
the router sets those time attributes for each packet object within its enqueuing and dequeuing methods.
The packet is then capable of reporting its latency (queueing delay) by computing the difference between its departure time and arrival time.

After the simulation, the collected information is used to calculate latency and throughput.
Average packet latency for a given source is computed as the sum of all of that source's packet's queueing delays, divided by the number of packets sent by that source.
Throughput for a given source is computed as the sum of all of that source's packet's lengths, divided by the difference between the last departure time and the first arrival time.
